%%%%%%%%%%%%%%%%%%%%%%%%%%%%%%%%%%%%%%%%%
% Twenty Seconds Resume/CV
% LaTeX Template
% Version 1.1 (8/1/17)
%
% This template has been downloaded from:
% http://www.LaTeXTemplates.com
%
% Original author:
% Carmine Spagnuolo (cspagnuolo@unisa.it) with major modifications by 
% Vel (vel@LaTeXTemplates.com)
%
% License:
% The MIT License (see included LICENSE file)
%
%%%%%%%%%%%%%%%%%%%%%%%%%%%%%%%%%%%%%%%%%

%----------------------------------------------------------------------------------------
%	PACKAGES AND OTHER DOCUMENT CONFIGURATIONS
%----------------------------------------------------------------------------------------

\documentclass[letterpaper]{twentysecondcv} % a4paper for A4
\usepackage[english, russian]{babel}
\usepackage[utf8]{inputenc}
\usepackage{lmodern}
%----------------------------------------------------------------------------------------
%	 PERSONAL INFORMATION
%----------------------------------------------------------------------------------------

% If you don't need one or more of the below, just remove the content leaving the command, e.g. \cvnumberphone{}

\profilepic{photo.jpeg} % Profile picture

\cvname{Karimov Aynur} % Your name
\cvjobtitle{Data Scientist} % Job title/career

\cvdate{5 Nov 1989} % Date of birth
\cvaddress{Russia} % Short address/location, use \newline if more than 1 line is required
\cvnumberphone{+7 9656166348} % Phone number %\cvnumberfone{+7 9637654435} % Phone number
\cvsite{} % Personal website
\cvmail{ainureg@ya.ru} % Email address
\education{
    \textbf{MSc, ФАКT} \\
    MIPT (МФТИ) \\
    2022 | Moscow\\

    \textbf{BMath, Кафедра теории функции и приближения}  \\
    КФУ Институт Механики и Математики \\
    2014 | Kazan, Russia
}
%----------------------------------------------------------------------------------------

\begin{document}

%----------------------------------------------------------------------------------------
%	 ABOUT ME
%----------------------------------------------------------------------------------------

\aboutme{MIPT alumni. I prefer opensource, always involved in self- and professional development.} % To have no About Me section, just remove all the text and leave \aboutme{}

%----------------------------------------------------------------------------------------
%	 SKILLS
%----------------------------------------------------------------------------------------

% Skill bar section, each skill must have a value between 0 an 6 (float)
\skills{{Python * Linux * \LaTeX /5.5},{SQL * R * C++/3}, {MLOps / 4 }, {computer vision/ 5}}

%------------------------------------------------

% Skill text section, each skill must have a value between 0 an 6
\skillstext{{Russian/C2},{English/B2},{Tatar/B2},{Spanish/A1},{Turkish/B1}}

%----------------------------------------------------------------------------------------
% \skillstext{ {git * dvc * linux * bash * opencv * pytorch}}


\makeprofile % Print the sidebar


\iffalse
%----------------------------------------------------------------------------------------
%	 EDUCATION
%----------------------------------------------------------------------------------------

\section{123}

\begin{twenty} % Environment for a list with descriptions
%	\twentyitem{since 1865}{Ph.D. {\normalfont candidate in Computer Science}}{Wonderland}{\emph{A Quantified Theory of Social Cohesion.}}
	
	%\twentyitem{<dates>}{<title>}{<location>}{<description>}
\end{twenty}
\fi
%----------------------------------------------------------------------------------------
%	 PUBLICATIONS
%----------------------------------------------------------------------------------------

\iffalse
\section{Publications}

\begin{twentyshort} % Environment for a short list with no descriptions
	\twentyitemshort{1865}{Chapter One, Down the Rabbit Hole.}
	\twentyitemshort{1865}{Chapter Two, The Pool of Tears.}
	\twentyitemshort{1865}{Chapter Three,  The Caucus Race and a Long Tale.}
	\twentyitemshort{1865}{Chapter Four,  The Rabbit Sends a Little Bill.}
	\twentyitemshort{1865}{Chapter Five,  Advice from a Caterpillar.}
	%\twentyitemshort{<dates>}{<title/description>}
\end{twentyshort}


%----------------------------------------------------------------------------------------
%	 AWARDS
%----------------------------------------------------------------------------------------

\section{Awards}

\begin{twentyshort} % Environment for a short list with no descriptions
	\twentyitemshort{1987}{All-Time Best Fantasy Novel.}
	\twentyitemshort{1998}{All-Time Best Fantasy Novel before 1990.}
	%\twentyitemshort{<dates>}{<title/description>}
\end{twentyshort}
\fi
%----------------------------------------------------------------------------------------
%	 EXPERIENCE
%----------------------------------------------------------------------------------------

\section{Experience}

\begin{twenty} % Environment for a list with descriptions
        \twentyitem
        {Feb 2022 —}
        {Now}
        {Machine Learning Engineer}
        {  \href{https://https://innopolis.university/en/}{Innopolis Universiy}}
        {}
        {
        {\begin{itemize}
            \item tasks:
            {\begin{itemize}
                \item semantic segmentation of medical data
            \end{itemize}}
            \item instruments:
            {\begin{itemize}
                \item torch, python, tfs
            \end{itemize}
            }

        \end{itemize}}
        }
        \\

        \twentyitem
        {Jul 2019 —}
        {Feb 2022}
        {Data scientst, comp. vision}
        {  \href{https://avtodoria.ru/}{Avtodoria}}
        {}
        {
        {\begin{itemize}
            \item tasks:
            {\begin{itemize}
                \item deep learning models (semantic segmentation) from scratch to prod
                \item applications with HD map and geodata
                \item image and geodata processing
            \end{itemize}}
            \item instruments:
            {\begin{itemize}
                \item python, bash, git, docker, pytest
                \item opencv, torch, geodandas etc.
            \end{itemize}
            }
        \end{itemize}}
        }
        \\
        \twentyitem
        {Sep 2017 —}
        {Oct 2018}
        {Data Scientist}
        {\href{http://maxima.life/}{Maxima}}
        {}
        {
        {\begin{itemize}
            \item tasks
                {\begin{itemize}
                    \item We were focused to create an application based on the bayesian statistics to replace people.
                    \item Statistical analysis of the data: finding and fitting distibutions, checking the hypotheses
                \end{itemize}
                }
            \item instruments: python, r , pandas, sklearn, statsmodels etc.
        \end{itemize}}
        }
\end{twenty}

%----------------------------------------------------------------------------------------
%	 OTHER INFORMATION
%----------------------------------------------------------------------------------------

\section{Events}

Main events since 2014

\begin{twenty}
        \twentyitem
        {Sep 2020 -}{Now}{Big data academy}
        {\href{http://data.mail.ru/}{mail.ru} }{student}
        {
            {\begin{itemize}
                \item learned subjects:
                    {\begin{itemize}
                    \item \textbf{advanced python}: pytest, logger, rest services
                    \item \textbf{introduction to algorythms and data structures}
                    \item \textbf{geodata analysis}: typical problems, geopy, geopandas
                    \item \textbf{machine learnong on graphs}: networkx, dgl
                    \item \textbf{machine learning in production}: git, docker, dvc, mlflow, airflow, kubernetes, cicd, reproducible code
                    \item \emph{advanced ml}: based on the bayesian statistics etc
                    \end{itemize}}

            \end{itemize}}
        }
        \\
        \twentyitem
        {2014, 2015}{}{Summer School}
        { \href{https://https://innopolis.university/en/}{Innopolis Universiy}}
        {}
        {
        {\begin{itemize}
            \item Courses of HPC (High Performance Computing)
            \item Сertificate of the professional development
        \end{itemize}}
        }
        \\
        \twentyitem
        {2014}{}{School for young specialists}
        {Intel}{}
        {
        {\begin{itemize}
            \item Courses of HPC, Intel programs for HPC
        \end{itemize}}
        }
        \\
        \twentyitem
        {2014}{}{Summer Bioinformatics School}
        {Bioinformatics Institute}{}
        {
        {\begin{itemize}
            \item Bioinformatics courses
        \end{itemize}}
        }
\end{twenty}


%----------------------------------------------------------------------------------------
%	 INTERESTS
%----------------------------------------------------------------------------------------

\section{Certificates}

\href{https://yadi.sk/d/aSyzcZ44-_l-8w?w=1}{Yandex.disk} folder for certificates

\section{Interests etc.}

Science, music, self-education, languages (speaking and programming)

\subsection{Future aims}

DevOps, reinforcement learning, ROS, SLAM


%----------------------------------------------------------------------------------------
%	 SECOND PAGE EXAMPLE
%----------------------------------------------------------------------------------------

%\newpage % Start a new page

%\makeprofile % Print the sidebar

%\section{Other information}

%\subsection{Review}

%Alice approaches Wonderland as an anthropologist, but maintains a strong sense of noblesse oblige that comes with her class status. She has confidence in her social position, education, and the Victorian virtue of good manners. Alice has a feeling of entitlement, particularly when comparing herself to Mabel, whom she declares has a ``poky little house," and no toys. Additionally, she flaunts her limited information base with anyone who will listen and becomes increasingly obsessed with the importance of good manners as she deals with the rude creatures of Wonderland. Alice maintains a superior attitude and behaves with solicitous indulgence toward those she believes are less privileged.

%\section{Other information}

%\subsection{Review}

%Alice approaches Wonderland as an anthropologist, but maintains a strong sense of noblesse oblige that comes with her class status. She has confidence in her social position, education, and the Victorian virtue of good manners. Alice has a feeling of entitlement, particularly when comparing herself to Mabel, whom she declares has a ``poky little house," and no toys. Additionally, she flaunts her limited information base with anyone who will listen and becomes increasingly obsessed with the importance of good manners as she deals with the rude creatures of Wonderland. Alice maintains a superior attitude and behaves with solicitous indulgence toward those she believes are less privileged.

%----------------------------------------------------------------------------------------




\end{document} 
